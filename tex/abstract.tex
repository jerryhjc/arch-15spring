% The Web has been a key mechanism behind globalization, fostering humanitarian and educational progress on a worldwide scale. However, Web applications have to be responsive to get used. The modern challenge for the Web is to deliver responsiveness to end-users, which has severe socio-economic consequences when not met. As a result, mobile Web browsing has become a hot research area. While mobile Web browser have begun to receive a lot of attention, their characterization and evaluation are ad-hoc. Our project aims to understand the state of methodologies for studying the mobile Web.

% Our results suggests that there is considerable room for improvement with current mobile Web browser study methodologies. According to our Webpage select criteria, the Webpages used in prior work exhibits severe redundancy. Their selections are either orders of magnitude more excessive than they need to be (e.g. studying thousands of pages when only tens are needed) or do not provide coverage of different Webpage behaviors. We also looking at hosting methodologies and discover that no approach is able to capture how a Webpage load looks when it is taken``live'' from the Web server. We hope our work motivates the need for a standardized and rigorous Webpage benchmark evaluation.



With process scaling and the demands of larger memory capacity in high-performance computing systems, DRAM technology has limit in leakage power and device capacity. Non Volatile Memory (NVM) has been widely studied as a potential DRAM alternative. However, large write power, high write latency, as well as reliability issue resulted from the resistance drift, bring in challenges for NVM based memory design. In this report, we try to apply several optimization techniques in NVM system, such as  cache replacement policy study, reducing unnecessary cache write back, L3 cache integration and hybrid memory study. Also, we try to simulate the system with gem5 tool and analyze with McPAT. Our results suggests that there is room for improvement in NVM system with optimized technique. What's more, we plan to do deeper research in these optimization techniques in this summer. 











